\documentclass[10pt]{article}

% Paketi koji će se koristiti
\usepackage[a4paper, landscape, margin=2.00cm]{geometry}
\usepackage{amsfonts, amssymb, amsmath}
\usepackage{graphicx}
\usepackage{float}
\usepackage{multirow}
\usepackage{array}
\usepackage[utf8]{inputenc}
\usepackage[T1]{fontenc}
\usepackage{fancyhdr}
\usepackage{lastpage}
\usepackage[dvipsnames]{xcolor}
\usepackage{listings}

% Dodaj header i footer
\pagestyle{fancy}
\renewcommand{\headrulewidth}{0pt}
\fancyhf{}
\setlength{\headheight}{32.15pt}
\lhead{}
\rhead{\includegraphics[height=1.00cm]{./tmux.png}}
\setlength{\footskip}{12.00pt}
\lfoot{}
\rfoot{str. {\thepage}/{\pageref{LastPage}}}

\begin{document}

\renewcommand{\arraystretch}{1.50}

    \begin{itemize}
        \item \texttt{tmux} je organiziran na način da je unutar jednog \textbf{Session}-a moguće napraviti više \textbf{Window}-a od kojih svaki sadrži više \textbf{Pane}-ova.
        \item \textit{defaultni} \texttt{tmux} prefix je \texttt{ctrl+b}. U konfiguracijskoj datoteci niže, zamjenjen je s \texttt{ctrl+x}. 
        \item u \textit{status bar}-u u podnožju \texttt{tmux} sjednice, nalaze se: ime sjednice (ili broj), broj prozora, ime prozora (ili \textit{shell}), \textit{hostname} i datum
    \end{itemize}

    % Sessions
    \section*{\color{ForestGreen} Sessions}
    \begin{tabular}{|>{\tt}p{9.00cm}|>{}p{15.50cm}|}
        \hline
        tmux list-sessions                          &   lista otvorenih ssh/tmux sjednica                   \\ \hline
        tmux [new-session -s <"ime-sjednice">]      &   pokretanje tmux sjednice [imena "ime-sjednice"]     \\ \hline
        ctrl+b,\$                                   &   preimenovanje sjednice                              \\ \hline
        ctrl+b,s                                    &   promjena/\textit{switch} sjednice                   \\ \hline
        ctrl+b,d                                    &   \textit{detach} s trenutne ssh/tmux sjednice        \\ \hline
        tmux attach-session [-t <"ime-sjednice" >]  &   \textit{attach} izgubljene/ugašene sjednice         \\ \hline
    \end{tabular}

    % Windows
    \section*{\color{ForestGreen} Windows}
    \begin{tabular}{|>{\tt}p{9.00cm}|>{}p{15.50cm}|}
        \hline
        ctrl+b,c                                    &   kreiranje novog prozora                             \\ \hline
        ctrl+b,p                                    &   prebacivanje u prethodni prozor                     \\ \hline
        ctrl+b,n                                    &   prebacivanje u slijedeći prozor                     \\ \hline
        ctrl+b,\&                                   &   gašenje prozora                                     \\ \hline
        ctrl+b,,                                    &   preimenovanje prozora                               \\ \hline
    \end{tabular}

    % Panes
    \section*{\color{ForestGreen} Panes}
    \begin{tabular}{|>{\tt}p{9.00cm}|>{}p{15.50cm}|}
        \hline
        ctrl+b,"                                    &   horizontalna podjela                                \\ \hline
        ctrl+b,\%                                   &   vertikalna podjela                                  \\ \hline
        ctrl+b,x exit ctrl+d                        &   gašenje \textit{pane}-a                             \\ \hline
        ctrl+b,<arrow>                              &   navigacija kroz \textit{pane}-ove                   \\ \hline
        ctrl+b,pustiti b,<arrow>                    &   promjena veličine \textit{pane}-a                   \\ \hline
        ctrl+b,z                                    &   \textit{toggle} \textit{zoomiranja} \textit{pane}-a (fokusira ga)     \\ \hline
    \end{tabular}

    % Copy mode
    \section*{\color{ForestGreen} Copy mode}
    \begin{itemize}
        \item u konfiguracijskoj datoteci niže postavljena je \textit{Vi style} navigacija kroz \textit{copy mode}
    \end{itemize}
    \begin{tabular}{|>{\tt}p{9.00cm}|>{}p{15.50cm}|}
        \hline
        ctrl+b,[                                    &   ulazak u \textit{copy mode}                                                 \\ \hline
        <space>                                     &   početak označavanja teksta                                                  \\ \hline
        <esc>                                       &   odustajanje / izlazak iz \textit{copy mode}-a                               \\ \hline
        <enter>                                     &   kraj označavanja teksta (\textit{copy}) i izlazak iz \textit{copy mode}-a   \\ \hline
        ctrl+b,]                                    &   \textit{paste}                                                              \\ \hline
    \end{tabular}

    % Customization
    \section*{\color{ForestGreen} Customization}
    \begin{itemize}
        \item \textit{tmux} konfiguracijska datoteka nalazi se u \texttt{$\sim$/.tmux.conf}
    \end{itemize}
    \texttt{
        \footnotesize
        \lstinputlisting[language=bash]{.tmux.conf}
    }

\end{document}